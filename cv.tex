%80%%%%%%%%%%%%%%%%%%%%%%%%%%%%%%%%%%%%%%%%%%%%%%%%%%%%%%%%%%%%%%%%%%%%%%%%%%%%%
% Here's my CV. It's a bit of a mess, but of course it is because I chose to
% use LaTeX to write it. I originally based it on a template found here:
%     http://texblog.org/2012/04/25/writing-a-cv-in-latex/
% I have gradually tweaked things until it no longer resembles that example.
% The document is essentially a series of tables with custom macros to write
% cells how I want. Each cell is a parbox which allows for newlines, itemized
% lists, etc.
\documentclass[10pt]{article}

\usepackage[margin=0.5in,bmargin=0.85in]{geometry}
\usepackage{array}
\usepackage{xcolor}
\usepackage{titlesec}
\usepackage{longtable}
\usepackage{hyperref}
\usepackage{enumitem}
\usepackage{fancyhdr}
%\usepackage{lastpage}

%%%%%%%%%%%%%%%%%%%%%%%%%%%%%%%%%%%%%%%%%%%%%%%%%%
% Colors, simple customizations
%%%%%%%%%%%%%%%%%%%%%%%%%%%%%%%%%%%%%%%%%%%%%%%%%%

\definecolor{lightgray}{gray}{0.8}
\definecolor{darkgray}{gray}{0.3}
\definecolor{urlcolor}{HTML}{0a606c}

\hypersetup{colorlinks=true, urlcolor=urlcolor}

% section style (small caps, underlined)
\titleformat{\section}
    {\normalsize}
    {\thesection}
    {1em}
    {\textsc}
    [\color{lightgray}\titlerule]

% subsection style
\titleformat*{\subsection}{\normalsize\bfseries}

% get rid of space surrounding/between items in itemize environment
\setlist[itemize]{noitemsep,nolistsep,leftmargin=*}
% make itemized list bullets smaller
\renewcommand\labelitemi{\footnotesize$\bullet$}

%%%%%%%%%%%%%%%%%%%%%%%%%%%%%%%%%%%%%%%%%%%%%%%%%%
% Table customization
%%%%%%%%%%%%%%%%%%%%%%%%%%%%%%%%%%%%%%%%%%%%%%%%%%

% Left table cell. Display a bold title with a date range below.
\newcommand\LColRaw[3]{\parbox[t]{#1}{
    \raggedleft%
    {\bf#2}\\
    {\small\color{darkgray}#3}}
}
\newcommand\LCol[2]{\LColRaw{1.3in}{#1}{#2}}

% Right table cell where you specify all formatting. Line breaks are ok.
\newcommand\RCol[1]{\parbox[t]{6in}{#1}}
% Right table cell with a title, subtitle, then remaining content
\newcommand\RColFancy[3]{\RCol{\textbf{#1} --- {\color{darkgray}#2}\\#3}}
\newcommand\RColList[3]{\RCol{\textbf{#1} --- {\color{darkgray}#2}#3}}

% override default spacing between table rows
\renewcommand{\arraystretch}{1.7}
% override default spacing between table columns
\renewcommand{\tabcolsep}{3pt}

%%%%%%%%%%%%%%%%%%%%%%%%%%%%%%%%%%%%%%%%%%%%%%%%%%
% Header/footer
%%%%%%%%%%%%%%%%%%%%%%%%%%%%%%%%%%%%%%%%%%%%%%%%%%
\pagestyle{fancy}
% clear the header and footer
\fancyhf{}
% get rid of header line
\renewcommand{\headrulewidth}{0pt}
% page number
%\cfoot{\thepage~of~\pageref{LastPage}}
% last edited date
%\rfoot{\footnotesize{Last Updated: \today}}



\begin{document}

\begin{center}
{\large Kenneth R. Lyons}
\end{center}

% left side contact info
\hspace*{-\parindent}%
\begin{minipage}[ht]{0.6\textwidth}
\begin{flushleft}
    Systems Engineer\\
    Systron Donner Inertial
\end{flushleft}
\end{minipage}
% right side contact info
\begin{minipage}[ht]{0.4\textwidth}
    \raggedleft{%
    \texttt{ixjlyons@gmail.com}\\
    \href{https://ixjlyons.com}{\nolinkurl{ixjlyons.com}}}
\end{minipage}


\section*{Education}

\vspace*{-\baselineskip}
\begin{longtable}{rl}
    \LCol{PhD}{2012--2018}
    & \RColFancy%
        {University of California, Davis}
        {Davis, CA}
        {Mechanical and Aerospace Engineering\\
         Advisor: Sanjay S. Joshi\\
         Dissertation: Human and Machine Learning in Myoelectric Control}\\
    \LCol{BS}{2008--2012}
    & \RColFancy%
        {University of California, Davis}
        {Davis, CA}
        {Mechanical Engineering\\Minor in Linguistics, graduation with high
         honors}
\end{longtable}

\section*{Core Skills}

\vspace*{-\baselineskip}
\begin{longtable}{cc}
    \LCol{Scientific Computing}{}
    & \RCol%
        {I have about 10 years of experience with Python for scientific
        computing: signal processing, machine learning, statistical analysis,
        and visualization. I'm a co-maintainer of pyqtgraph, a widely-used
        high-performance data visualization library, so I keep up with best
        practices for packaging and reproducible environments.}\\
    \LCol{Application Design}{}
    & \RCol%
        {I've developed graphical user interfaces with Python (PyQt), MATLAB,
        C++ (Qt), and Java (Android), predominantly for building human-computer
        interface experiments and demonstrations. I've also built interfaces
        for exploring and presenting data.}\\
    \LCol{Embedded Systems}{}
    & \RCol%
        {I have experience developing bare-metal C code for AVR and some ARM
        Cortex (TI, ST) microcontrollers as well as TI DSPs. I enjoy all parts
        of the development process: system design, schematic capture and PCB
        layout, programming, and verification.}\\
    \LCol{Experiment Design}{}
    & \RCol%
        {My primary experience with experiment design is for human subject
        studies for research applications, but the analytical techniques apply
        widely. I enjoy taking an existing dataset and answering the question:
        what's the next step?}\\
\end{longtable}


\section*{Experience}

\vspace*{-\baselineskip}
\begin{longtable}{cc}
    \LCol{INS/GPS Engineer}{2018--present}
    & \RColList%
        {Systron Donner Inertial}
        {Concord, CA}
        {\begin{itemize}
            \item Developed models of signal processing pipelines for strapdown
            inertial navigation systems based on quartz MEMS gyroscopes.
            \item Implemented processing algorithms in C for Texas Instruments
            DSPs as well as VHDL for Xilinx FPGAs.
            \item Developed and performed tests for inertial navigation system
            performance verification.
            \item Took on FPGA engineer role to complete DO-254 hardware and
            simulation verification efforts for a project after the company
            faced several key departures.
            \item Learned and followed AS9100 quality management processes.
         \end{itemize}}\\
    \LCol{Graduate Student Researcher}{2012--2018}
    & \RColList%
        {Robotics, Autonomous Systems, and Controls Lab}
        {UC Davis}
        {\begin{itemize}
            \item Implemented signal processing and machine learning algorithms
            for decoding user intentions from surface electromyography (EMG) in
            real time on both desktop computer systems and the Android mobile
            platform.
            \item Designed software frameworks to control robotics simulations
            and computer interfaces via electrophysiological signals.
            \item Designed human subject experiments for evaluating myoelectric
            prosthesis control techniques and human motor adaptation.
            \item Analyzed research data and synthesized findings into
            publications and conference presentations.
            \item Designed and built printed circuit boards for powering EMG
            sensors and connecting them to mobile devices and data acquisition
            systems.
         \end{itemize}}\\
    \LCol{Teaching Assistant}{2012--2017}
    & \RColList%
        {Department of Mechanical and Aerospace Engineering}
        {UC Davis}
        {\begin{itemize}
            \item Supervised experimental methods lab sessions and graded lab
            reports (EME 107B).
            \item Helped develop a computational framework for learning
            mechanical vibrations via Python, created homework assignments,
            and administered a JupyterHub for students to use (ENG 122).
            \item Coached mechanical engineering senior design teams and
            produced a workshop on implementing control systems with Arduino
            (EME185).
         \end{itemize}}\\
    \LCol{Undergraduate Researcher}{2011--2012}
    & \RColList%
        {Sports Biomechanics Lab, Integration Engineering Lab}
        {UC Davis}
        {\begin{itemize}
            \item Assisted a PhD student with a variety of electrical and
            programming tasks for his robotic riderless bicycle dissertation
            project.
            \item Developed and implemented an algorithm for steering encoder
            calibration.
            \item Developed a new STEM outreach program for middle and high
            school students to get hands-on experience with programming and
            robotics.
         \end{itemize}}\\
    \LCol{Tutor}{2010--2012}
    & \RColList%
        {Superb Tutors}
        {Davis, CA}
        {\begin{itemize}
            \item Tutored high school and college students in math, physics,
            and engineering courses.
         \end{itemize}}
\end{longtable}


\section*{Selected Projects}

\vspace*{-\baselineskip}
\begin{longtable}{cc}
    \LCol{AxoPy}{2016--present}
    & \RCol%
        {Python library for rapidly developing human-computer interface
        experiments based on electrophysiological signals for control. Includes
        functionality for data acquisition, data storage, signal processing,
        intention estimation, and graphical user interface feedback.\\
        \url{https://github.com/ucdrascal/axopy}}\\
    \LCol{Resonance}{2016--2018}
    & \RCol%
        {Open course materials for undergraduate mechanical vibrations,
        including Jupyter notebooks for learning and practicing analyses and
        a Python library for working with vibratory systems. This was
        a collaborative effort with the instructor for the class at UCD.\\
        \url{https://github.com/moorepants/resonance}}\\
    \LCol{Walk Again}{2013--2014}
    & \RCol%
        {International collaborative effort to produce a brain-controlled
        exoskeleton demonstrated at the 2014 FIFA World Cup. I worked as a part
        of the human-machine interface team and created a LED-based feedback
        system to enable robust control during the demonstration.}\\
    \LCol{SecondEyes}{2011--2015}
    & \RCol%
        {A telepresence mobile robot built for individuals with severe mobility
        impairments to virtually view their surroundings using a single
        electromyography (EMG) sensor. I wrote the Android application to
        control the robot and developed the robot's electronics and
        firmware.}\\
\end{longtable}


\section*{Selected Publications}

\vspace*{-\baselineskip}
\begin{longtable}{cc}
    % AxoPy JOSS
    \LCol{2019}{}
    & \RCol{%
        \textbf{K. R. Lyons} and B. W. L. Margolis,
        ``AxoPy: A Python Library for Implementing Human-Computer Interface
            Experiments,''
        \emph{Journal of Open Source Software}, vol. 4, no. 34, 2019.}\\
    % EMBC 2018
    \LCol{2018}{}
    & \RCol{%
        \textbf{K. R. Lyons} and S. S. Joshi,
        ``Effects of Mapping Uncertainty on Visuomotor Adaptation to
            Trial-By-Trial Perturbations with Proportional Myoelectric
            Control,''
        \emph{Proceedings of the IEEE Engineering in Medicine and Biology
            Society Conference (EMBC)},
        Honolulu, HI,
        2018.}\\
    % TNSRE
    \LCol{2018}{}
    & \RCol{%
        \textbf{K. R. Lyons} and S. S. Joshi,
        ``Upper Limb Prosthesis Control for High-Level Amputees via Myoelectric
            Recognition of Leg Gestures,''
        \emph{IEEE Transactions on Neural Systems and Rehabilitation
            Engineering}, vol. 26, no. 4, 2018.}\\
    % Human Movement Science
    \LCol{2016}{}
    & \RCol{%
        I. M. Skavhaug, \textbf{K. R. Lyons}, A. Nemchuk, S. Muroff, and S.
            Joshi,
        ``Learning to Modulate the Partial Powers of a Single sEMG Power
            Spectrum Through a Novel Human-Computer Interface,''
        \emph{Human Movement Science}, vol. 47, pp. 60--69,
        2016.}\\
    % SFN 2014 (Walk Again a)
    \LCol{2014}{}
    & \RCol{%
        A. Lin, D. Schwarz, R. Sellaouti, S. Shokur, R. C. Moioli, F. L. Brasil,
            K. R. Fast, N. A. Peretti, A. Takigami, S. Gallo, \textbf{K. R.
            Lyons}, P. Mittendorfer, M. Lebedev, S. Joshi, G. Cheng, E. Morya,
            A. Rudolph, M. Nicolelis,
        ``The Walk Again Project: Brain-Controlled Exoskeleton Locomotion,''
        \emph{Annual Meeting of the Society for Neuroscience (SfN)},
        Washington D.C.,
        2014.}\\
    % ICORR 2013
    \LCol{2013}{}
    & \RCol{%
        \textbf{K. R. Lyons} and S. S. Joshi,
        ``Paralyzed Subject Controls Telepresence Mobile Robot Using Novel
            {sEMG} Brain-Computer Interface: Case Study,''
        \emph{Proceedings of the IEEE International Conference on
            Rehabilitation Robotics (ICORR)},
        Seattle, WA,
        2013.}\\
    \LCol{\hfill}{}
    & \RCol{\footnotesize{Full list at
        \href{https://ixjlyons.com\#publications}%
             {\nolinkurl{ixjlyons.com\#publications}}}}\\
\end{longtable}

\end{document}
%80%%%%%%%%%%%%%%%%%%%%%%%%%%%%%%%%%%%%%%%%%%%%%%%%%%%%%%%%%%%%%%%%%%%%%%%%%%%%%
