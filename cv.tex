%80%%%%%%%%%%%%%%%%%%%%%%%%%%%%%%%%%%%%%%%%%%%%%%%%%%%%%%%%%%%%%%%%%%%%%%%%%%%%%
% Here's my CV. It's a bit of a mess, but of course it is because I chose to
% use LaTeX to write it. I originally based it on a template found here:
%     http://texblog.org/2012/04/25/writing-a-cv-in-latex/
% I have gradually tweaked things until it no longer resembles that example.
% The document is essentially a series of tables with custom macros to write
% cells how I want. Each cell is a parbox which allows for newlines, itemized
% lists, etc.
\documentclass[10pt]{article}

\usepackage[margin=0.5in]{geometry}
\usepackage{array}
\usepackage{xcolor}
\usepackage{titlesec}
\usepackage{longtable}
\usepackage{hyperref}
\usepackage{enumitem}

%%%%%%%%%%%%%%%%%%%%%%%%%%%%%%%%%%%%%%%%%%%%%%%%%%
% Colors, simple customizations
%%%%%%%%%%%%%%%%%%%%%%%%%%%%%%%%%%%%%%%%%%%%%%%%%%

\definecolor{lightgray}{gray}{0.8}
\definecolor{darkgray}{gray}{0.3}
\definecolor{urlcolor}{HTML}{0a606c}

\hypersetup{colorlinks=true, urlcolor=urlcolor}

% section style (small caps, underlined)
\titleformat{\section}
    {\normalsize}
    {\thesection}
    {1em}
    {\textsc}
    [\color{lightgray}\titlerule]

% subsection style
\titleformat*{\subsection}{\normalsize\bfseries}

% get rid of space surrounding/between items in itemize environment
\setlist[itemize]{noitemsep,nolistsep,leftmargin=*}
% make itemized list bullets smaller
\renewcommand\labelitemi{\footnotesize$\bullet$}

%%%%%%%%%%%%%%%%%%%%%%%%%%%%%%%%%%%%%%%%%%%%%%%%%%
% Table customization
%%%%%%%%%%%%%%%%%%%%%%%%%%%%%%%%%%%%%%%%%%%%%%%%%%

% Left table cell. Display a bold title with a date range below.
\newcommand\LColRaw[3]{\parbox[t]{#1}{
    \raggedleft%
    {\bf#2}\\
    {\small\color{darkgray}#3}}
}
\newcommand\LCol[2]{\LColRaw{1.3in}{#1}{#2}}

% Right table cell where you specify all formatting. Line breaks are ok.
\newcommand\RCol[1]{\parbox[t]{6in}{#1}}
% Right table cell with a title, subtitle, then remaining content
\newcommand\RColFancy[3]{\RCol{\textbf{#1} --- {\color{darkgray}#2}\\#3}}
\newcommand\RColList[3]{\RCol{\textbf{#1} --- {\color{darkgray}#2}#3}}

% override default spacing between table rows
\renewcommand{\arraystretch}{1.7}
% override default spacing between table columns
\renewcommand{\tabcolsep}{3pt}


\begin{document}

\thispagestyle{empty}
\pagestyle{empty}

\begin{center}
{\large Kenneth R. Lyons}
\end{center}

% left side contact info
\hspace*{-\parindent}%
\begin{minipage}[ht]{0.6\textwidth}
\begin{flushleft}
    PhD Candidate\\
    Mechanical and Aerospace Engineering\\
    University of California, Davis
\end{flushleft}
\end{minipage}
% right side contact info
\begin{minipage}[ht]{0.4\textwidth}
    \raggedleft{%
    \texttt{ixjlyons@gmail.com}\\
    \href{http://ixjlyons.com}{\nolinkurl{ixjlyons.com}}}
\end{minipage}


\section*{Education}

\vspace*{-\baselineskip}
\begin{longtable}{rl}
    \LCol{PhD}{2012--present}
        & \RColFancy%
            {University of California, Davis}
            {Davis, CA}
            {Mechanical and Aerospace Engineering\\Advisor: Sanjay S. Joshi}
    \\
    \LCol{BS}{2008--2012}
        & \RColFancy%
            {University of California, Davis}
            {Davis, CA}
            {Mechanical Engineering, minor in Linguistics, graduation with high
            honors}
\end{longtable}


\section*{Experience}

\vspace*{-\baselineskip}
\begin{longtable}{cc}
    \LCol{Graduate Student Researcher}{2012--present}
    & \RColList%
        {Robotics, Autonomous Systems, and Controls Lab}
        {UC Davis}
        {\begin{itemize}
            \item Developed Android applications for several human-computer
            interface studies based on surface electromyography (EMG).
            \item Created printed circuit boards for powering EMG sensors and
            connecting them to mobile devices and data acquisition systems.
            \item Assisted with experiment design, data analysis, and
            manuscript preparation for lab publications.
         \end{itemize}}\\
    \LCol{Teaching Assistant}{2012--2017}
    & \RColList%
        {Department of Mechanical and Aerospace Engineering}
        {UC Davis}
        {\begin{itemize}
            \item Supervised experimental methods lab sessions and graded lab
            reports (EME 107B).
            \item Helped develop a computational framework for learning
            mechanical vibrations via Python, created homework assignments,
            and administered a JupyterHub for students to use (ENG 122).
            \item Coached mechanical engineering senior design teams and
            produced a workshop on implementing control systems with Arduino
            (EME185).
         \end{itemize}}\\
    \LCol{Undergraduate Researcher}{Winter 2012}
    & \RColList%
        {Sports Biomechanics Lab}
        {UC Davis}
        {\begin{itemize}
            \item Assisted a PhD student with his robotic (riderless) bicycle
            dissertation project.
            \item Produced a wiring diagram for the robot's circuitry and
            hardware.
            \item Developed a C program for the robot's STM32 microcontroller
            for steering encoder calibration.
         \end{itemize}}\\
    \LCol{Undergraduate Researcher}{Summer 2011}
    & \RColList%
        {Integration Engineering Lab}
        {UC Davis}
        {\begin{itemize}
            \item Developed problems and challenges for programming and
            robotics competitions for middle and high school students, which
            later became part of C-STEM Day at UC Davis.
         \end{itemize}}\\
    \LCol{Tutor}{2010--2012}
    & \RColList%
        {Superb Tutors}
        {Davis, CA}
        {\begin{itemize}
            \item Tutored high school and college students in math, physics,
            and engineering courses.
         \end{itemize}}
\end{longtable}


\section*{Projects}

\vspace*{-\baselineskip}
\begin{longtable}{cc}
    \LCol{AxoPy}{2016--present}
        & \RCol%
            {Python library for rapidly developing human-computer interface
            experiments based on electrophysiological signals. Includes
            functionality for data acquisition, data storage, signal
            processing, intention estimation, and graphical user interface
            feedback.\\
            \url{https://github.com/ucdrascal/axopy}}\\
    \LCol{Resonance}{2017--present}
        & \RCol%
            {Open course materials for undergraduate mechanical vibrations,
            including Jupyter notebooks for learning and practicing analyses
            and a Python library for working with vibratory systems.\\
            \url{https://github.com/moorepants/resonance}}\\
    \LCol{PyGesture}{2014--2016}
        & \RCol%
            {Myoelectric gesture recognition suite for prosthesis control
            experiments, written in Python. Pre-cursor to AxoPy.\\
            \url{https://github.com/ixjlyons/pygesture}}\\
    \LCol{Walk Again}{2013--2014}
        & \RCol%
            {International collaborative effort to produce a brain-controlled
            exoskeleton demonstrated at the 2014 FIFA World Cup. I worked as
            a part of the human-machine interface team and created a LED-based
            feedback system to enable robust control during the
            demonstration.}\\
    \LCol{SecondEyes}{2011--2015}
        & \RCol%
            {A telepresence mobile robot built for individuals with severe
            mobility impairments to virtually view their surroundings using
            a single electromyography (EMG) sensor. I wrote the Android
            application to control the robot and developed the robot's
            electronics and firmware.}\\
\end{longtable}


\section*{Awards and Recognition}

\vspace*{-\baselineskip}
\begin{longtable}{cc}
    \LCol{Service Award}{2012}
        & \RColFancy%
            {Department of Mechanical and Aerospace Engineering}
            {UC Davis}
            {Award for participation in the Computing and Robotics Outreach
            as part of my work with the Integration Engineering Lab.}\\
    \LCol{Summer Camp Award}{2012}
        & \RColFancy%
            {Clinical and Translational Science Center}
            {UC Davis}
            {Grant used to fund development of the electromyography-based
            controller for the SecondEyes telepresence mobile robot during the
            summer before starting my PhD program.}
\end{longtable}


\section*{Other Interests}

\vspace*{-\baselineskip}
\begin{longtable}{cc}
    \LCol{Python}{}
        & \RCol{Python is my main programming language and I have used it in
        a variety of contexts since 2010. In 2017, I attended the SciPy
        conference in Austin, Texas and co-taught a tutorial session on
        automatic code generation with SymPy
        (\url{http://www.sympy.org/scipy-2017-codegen-tutorial}).}\\
    \LCol{Linux}{}
        & \RCol{I have been using Linux-based systems since 2008, and
        I currently serve as the secretary of one of the longest-running Linux
        Users' Groups (LUGOD).}\\
    \LCol{Free Software}{}
        & \RCol{I am an advocate of free (libre) and open source software and
        enjoy contributing to projects that I use. I have aimed to make most of
        my work as a PhD student available under permissive licenses.}
\end{longtable}


\section*{Publications}

\subsection*{Papers}

\vspace*{-\baselineskip}
\begin{longtable}{cc}
    % EMBC 2018
    \LCol{(submitted)}{} & \RCol{%
        \textbf{K. R. Lyons} and S. S. Joshi,
        ``Effects of Mapping Uncertainty on Visuomotor Adaptation to
            Trial-By-Trial Perturbations with Proportional Myoelectric
            Control,''
        \emph{Proceedings of the IEEE Engineering in Medicine and Biology
            Society Conference (EMBC)},
        2018.}\\
    % THMS
    \LCol{(submitted)}{} & \RCol{%
        I. M. Skavhaug, \textbf{K. R. Lyons}, H. Chen, L. Barry, B. Korte, S.
        S. Joshi,
        ``A Minimal Recording Configuration sEMG Human-Computer Interface for
            Cursor Control,''
        \emph{IEEE Transactions on Human-Machine Systems}.}\\
    % TNSRE
    \LCol{(in press)}{} & \RCol{%
        \textbf{K. R. Lyons} and S. S. Joshi,
        ``Upper Limb Prosthesis Control for High-Level Amputees via Myoelectric
            Recognition of Leg Gestures,''
        \emph{IEEE Transactions on Neural Systems and Rehabilitation
            Engineering}.}\\
    % EMBC 2016
    \LCol{2016}{} & \RCol{%
        \textbf{K. R. Lyons} and S. S. Joshi,
        ``Real-Time Evaluation of a Myoelectric Control Method for High-Level
            Upper Limb Amputees Based on Homologous Leg Movements,''
        \emph{Proceedings of the IEEE Engineering in Medicine and Biology
            Society Conference (EMBC)},
        Orlando, FL,
        2016.}\\
    % Human Movement Science
    \LCol{2016}{} & \RCol{%
        I. M. Skavhaug, \textbf{K. R. Lyons}, A. Nemchuk, S. Muroff, and S.
            Joshi,
        ``Learning to Modulate the Partial Powers of a Single sEMG Power
            Spectrum Through a Novel Human-Computer Interface,''
        \emph{Human Movement Science}, vol. 47, pp. 60--69,
        2016.}\\
    % RSS 2016 workshop paper
    \LCol{2016}{} & \RCol{%
        J. Varley, S. Sridhar, J. Weisz, E. Rand, \textbf{K. Lyons}, S. Joshi,
            J. Stein, and P. Allen,
        ``Human Robot Interface for Assistive Grasping,''
        \emph{Socially \& Physically Assistive Robotics for Humanity (workshop
            at Robotics: Science and Systems)},
        Ann Arbor, MI,
        2016.}\\
    % RESNA 2015
    \LCol{2015}{} & \RCol{%
        \textbf{K. R. Lyons} and S. S. Joshi,
        ``A Case Study on Classification of Foot Gestures via Surface
            Electromyography,''
        \emph{Annual Conference of the Rehabilitation Engineering and Assistive
            Technology Society of America (RESNA)},
        Denver, CO,
        2015.}\\
    % ICORR 2013
    \LCol{2013}{} & \RCol{%
        \textbf{K. R. Lyons} and S. S. Joshi,
        ``Paralyzed Subject Controls Telepresence Mobile Robot Using Novel
            {sEMG} Brain-Computer Interface: Case Study,''
        \emph{Proceedings of the IEEE International Conference on
            Rehabilitation Robotics (ICORR)},
        Seattle, WA,
        2013.}\\
\end{longtable}

\subsection*{Conference Posters and Abstracts}

\vspace*{-\baselineskip}
\begin{longtable}{cc}
    % EMBC 2016 (Maria's 1-page paper)
    \LCol{2016}{} & \RCol{%
        I. M. Skavhaug, \textbf{K. R. Lyons}, S. D. Muroff, H. Chen, L. Barry,
            B. Korte, and S. S. Joshi,
        ``Fitts' Law Evaluation of a Passive Rotation Paradigm for
            Two-Dimensional Cursor Control with a Single sEMG Signal''
        \emph{Proceedings of the IEEE Engineering in Medicine and Biology Society
            Conference (EMBC)},
        Orlando, FL,
        2016.}\\
    % SFN 2015 (my poster)
    \LCol{2015}{} & \RCol{%
        \textbf{K. R. Lyons} and S. S. Joshi,
        ``Real-Time Myoelectric Control of a Virtual Upper Limb Prosthesis via
            Lower Leg Gestures: Preliminary Results,''
        \emph{Annual Meeting of the Society for Neuroscience (SfN)},
        Chicago, IL,
        2015.}\\
    % SFN 2015 (Maria's poster)
    \LCol{2015}{} & \RCol{%
        I. M. Skavhaug, \textbf{K. R. Lyons}, A. Nemchuk, S. Muroff, and S. S.
            Joshi,
        ``Control of a Cursor in Two Dimensions with One Single sEMG Signal:
            Learning of a Novel Motor Skill,''
        \emph{Annual Meeting of the Society for Neuroscience (SfN)},
        Chicago, IL,
        2015.}\\
    % SFN 2014 (my poster)
    \LCol{2014}{} & \RCol{%
        \textbf{K. R. Lyons} and S. S. Joshi,
        ``Arm Prosthetic Control through Electromyographic Recognition of Leg
            Gestures,''
        \emph{Annual Meeting of the Society for Neuroscience (SfN)},
        Washington D.C.,
        2014.}\\
    % SFN 2014 (Maria's poster)
    \LCol{2014}{} & \RCol{%
        I. M. Skavhaug, C. Dao, \textbf{K. R. Lyons}, A. Powell, L. Davidson,
            S. Joshi,
        ``Use of an Ear-Mounted Myoelectric Human-Computer Interface in the
            Home: A Pediatric Case Study with Tetra-Amelia Syndrome Subject,''
        \emph{Annual Meeting of the Society for Neuroscience (SfN)},
        Washington D.C.,
        2014.}\\
    % SFN 2014 (Walk Again a)
    \LCol{2014}{} & \RCol{%
        A. Lin, D. Schwarz, R. Sellaouti, S. Shokur, R. C. Moioli, F. L. Brasil,
            K. R. Fast, N. A. Peretti, A. Takigami, S. Gallo, \textbf{K. R.
            Lyons}, P. Mittendorfer, M. Lebedev, S. Joshi, G. Cheng, E. Morya,
            A. Rudolph, M. Nicolelis,
        ``The Walk Again Project: Brain-Controlled Exoskeleton Locomotion,''
        \emph{Annual Meeting of the Society for Neuroscience (SfN)},
        Washington D.C.,
        2014.}\\
    % SFN 2014 (Walk Again b)
    \LCol{2014}{} & \RCol{%
        F. L. Brasil, R. C. Moioli, S. Shokur, K. Fast, A. L. Lin, N. A.
            Peretti, A. Takigami, \textbf{K. R. Lyons}, D. J. Zielinski, L.
            Sawaki, S. Joshi, E. Morya, M. A. L. Nicolelis,
        ``The Walk Again Project: An EEG/EMG Training Paradigm to Control
            Locomotion,''
        \emph{Annual Meeting of the Society for Neuroscience (SfN)},
        Washington D.C.,
        2014.}
\end{longtable}

\end{document}
%80%%%%%%%%%%%%%%%%%%%%%%%%%%%%%%%%%%%%%%%%%%%%%%%%%%%%%%%%%%%%%%%%%%%%%%%%%%%%%
