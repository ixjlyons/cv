%80%%%%%%%%%%%%%%%%%%%%%%%%%%%%%%%%%%%%%%%%%%%%%%%%%%%%%%%%%%%%%%%%%%%%%%%%%%%%%
% template originally found here:
%     http://texblog.org/2012/04/25/writing-a-cv-in-latex/
\documentclass[10pt]{article}

\usepackage{array}
\usepackage{xcolor}
\usepackage{bibentry}
\usepackage[margin=0.5in]{geometry}
\usepackage{sectsty}
\usepackage{makecell}
\usepackage{multirow}
\usepackage{tabularx}
\usepackage{titlesec}
\usepackage{longtable}
\usepackage{hyperref}

\usepackage[T1]{fontenc}
\usepackage{lmodern}

\definecolor{lightgray}{gray}{0.8}

\titlespacing\section{0pt}{1em}{0em}
\titleformat{\section}
    {\normalsize}
    {\thesection}
    {1em}
    {\MakeUppercase}
    [\color{lightgray}\titlerule]

% Left table cell. Display a bold title with a date range below.
\newcommand\LColRaw[3]{\parbox[t]{#1}{
    \raggedleft%
    {\bf#2}\\
    {\small\color{gray}#3}}
}
\newcommand\LCol[2]{\LColRaw{1.3in}{#1}{#2}}

% Right table cell where you specify all formatting. Line breaks are ok.
\newcommand\RCol[1]{\parbox[t]{6.0in}{#1}}
% Right table cell with a title, subtitle, then remaining content
\newcommand\RColFancy[3]{\RCol{\textbf{#1} --- #2\\#3}}

% override default table row spacing
\renewcommand{\arraystretch}{1.5}
% override default table col spacing
\renewcommand{\tabcolsep}{3pt}

\begin{document}

\thispagestyle{empty}
\pagestyle{empty}

% title (name)
\begin{center}
%\noindent
{\Large Kenneth R. Lyons}
\end{center}

\vspace{1em}

% left side contact info
\hspace*{-\parindent}%
\begin{minipage}[ht]{0.6\textwidth}
\begin{flushleft}
    PhD Candidate\\
    Robotics, Autonomous Systems, and Controls Laboratory\\
    Department of Mechanical and Aerospace Engineering\\
    University of California, Davis
\end{flushleft}
\end{minipage}
% right side contact info
\begin{minipage}[ht]{0.4\textwidth}
    \raggedleft{%
    \texttt{ixjlyons@gmail.com}\\
    \href{http://ixjlyons.com}{\nolinkurl{ixjlyons.com}}}
\end{minipage}


\section*{Education}

\begin{longtable}{rl}
    \LCol{PhD}{2012--2018}
        & \RColFancy%
            {University of California, Davis}
            {Davis, CA}
            {Mechanical and Aerospace Engineering\\Advisor: Sanjay S. Joshi}
    \\
    \LCol{BS}{2008--2012}
        & \RColFancy%
            {University of California, Davis}
            {Davis, CA}
            {Mechanical Engineering, minor in Linguistics, graduation with high
            honors}
\end{longtable}


%\section*{Awards and Recognition}
%\begin{tabular}{L!{\VRule}R}
%    2012
%        & UC Davis Department of Mechanical and Aerospace Engineering Service
%            Award, for participation in the Computing and Robotics Outreach
%            Club\\
%    [5pt]
%    2012
%        & UC Davis Clinical and Translational Science Center / College of
%            Engineering Summer Camp Program\\
%\end{tabular}

\section*{Experience}

\begin{longtable}{cc}
    \LCol{Teaching Assistant}{2012--present}
        & \RColFancy%
            {Department of Mechanical and Aerospace Engineering}
            {UC Davis}
            {Supervised laboratory sessions in experimental methods course and
            graded lab reports (EME 107B). Assisted with introducing scientific
            computing with Python for mechanical vibrations (ENG 122).}\\
    \LCol{Graduate Student Researcher}{2012--present}
        & \RColFancy%
            {Robotics, Autonomous Systems, and Controls Lab}
            {UC Davis}
            {Worked on computer and machine interface control using
            electromyography, including upper limb prosthetic control. Lead
            Android application development for additional laboratory
            experiments.}\\
    \LCol{Undergraduate Student Researcher}{Winter 2012}
        & \RColFancy%
            {Sports Biomechanics Lab}
            {UC Davis}
            {Assisted Ph.D. student with riderless bicycle project, including
            some embedded system analysis and literature review of bicycle
            dynamics.}\\
    \LCol{Undergraduate Student Researcher}{Summer 2011}
        & \RColFancy%
            {Integration Engineering Lab}
            {UC Davis}
            {Developed programming and robotics competitions for middle and
            high school students, which later became part of C-STEM Day at UC
            Davis.}\\
    \LCol{Tutor}{2010--2012}
        & \RColFancy%
            {Superb Tutors}
            {Davis, CA}
            {Tutored high school and college students in math, physics, and
            engineering courses.}
\end{longtable}

\section*{Projects}

\begin{longtable}{cc}
    \LCol{AxoPy}{2016--present}
        & \RCol%
            {Library for writing experiments using electrophysiological signals
            to control machine and computer interfaces.\\
            \url{https://github.com/ucdrascal/axopy}}\\
    \LCol{Resonance}{2017--present}
        & \RCol%
            {An open set of materials for an undergraduate course in mechanical
            vibrations. As a TA for the course these materials were developed
            for, I created homework assignments and administered a JupyterHub
            server for students to use.\\
            \url{https://github.com/moorepants/resonance}}\\
    \LCol{PyGesture}{2014--2016}
        & \RCol%
            {Open-source myoelectric gesture recognition suite for end-to-end
            prosthesis control experiments, written in Python. Includes data
            acquisition, signal processing, classification, graphical user
            interface, and communication with real-time simulation software.\\
            \url{https://github.com/ixjlyons/pygesture}}\\
    \LCol{Walk Again}{2013--2014}
        & \RCol%
            {International project which produced a brain-controlled
            exoskeleton demonstrated at the 2014 FIFA World Cup. Worked as
            a part of the human-machine interface team and created an LED-based
            feedback system to enable robust control during the
            demonstration.}\\
    \LCol{SecondEyes}{2011--2015}
        & \RCol%
            {SecondEyes is a telepresence mobile robot meant to allow
            individuals with severe mobility impairments to virtually view
            their surroundings via a single electromyographic (EMG) sensor.
            I was responsible for creating the Android application to control
            the robot and display its camera feed as well as developing the
            robot's electronics.}\\
\end{longtable}


\section*{Publications}

\begin{longtable}{cc}
    % EMBC 2018
    \LCol{(submitted)}{} & \RCol{%
        \textbf{K. R. Lyons} and S. S. Joshi,
        ``Effects of Mapping Uncertainty on Visuomotor Adaptation to
            Trial-By-Trial Perturbations with Proportional Myoelectric
            Control,''
        \emph{Proceedings of the IEEE Engineering in Medicine and Biology
            Society Conference (EMBC)},
        2018.}\\
    % THMS
    \LCol{(submitted)}{} & \RCol{%
        I. M. Skavhaug, \textbf{K. R. Lyons}, B. Korte, L. Barry, H. Chen, S.
            S. Joshi,
        ``A Minimal Recording Configuration sEMG Human-Computer Interface for
            Cursor Control,''
        \emph{IEEE Transactions on Human-Machine Systems}.}\\
    % TNSRE
    \LCol{(in press)}{} & \RCol{%
        \textbf{K. R. Lyons} and S. S. Joshi,
        ``Upper Limb Prosthesis Control for High-Level Amputees via Myoelectric
            Recognition of Leg Gestures,''
        \emph{IEEE Transactions on Neural Systems and Rehabilitation
            Engineering}.}\\
    % EMBC 2016
    \LCol{2016}{} & \RCol{%
        \textbf{K. R. Lyons} and S. S. Joshi,
        ``Real-Time Evaluation of a Myoelectric Control Method for High-Level
            Upper Limb Amputees Based on Homologous Leg Movements,''
        \emph{Proceedings of the IEEE Engineering in Medicine and Biology
            Society Conference (EMBC)},
        Orlando, FL,
        2016.}\\
    % Human Movement Science
    \LCol{2016}{} & \RCol{%
        I. M. Skavhaug, \textbf{K. R. Lyons}, A. Nemchuk, S. Muroff, and S.
            Joshi,
        ``Learning to Modulate the Partial Powers of a Single sEMG Power
            Spectrum Through a Novel Human-Computer Interface,''
        \emph{Human Movement Science}, vol. 47, pp. 60--69,
        2016.}\\
    % RSS 2016 workshop paper
    \LCol{2016}{} & \RCol{%
        J. Varley, S. Sridhar, J. Weisz, E. Rand, \textbf{K. Lyons}, S. Joshi,
            J. Stein, and P. Allen,
        ``Human Robot Interface for Assistive Grasping,''
        \emph{Socially \& Physically Assistive Robotics for Humanity (workshop
            at Robotics: Science and Systems)},
        Ann Arbor, MI,
        2016.}\\
    % RESNA 2015
    \LCol{2015}{} & \RCol{%
        \textbf{K. R. Lyons} and S. S. Joshi,
        ``A Case Study on Classification of Foot Gestures via Surface
            Electromyography,''
        \emph{Annual Conference of the Rehabilitation Engineering and Assistive
            Technology Society of America (RESNA)},
        Denver, CO,
        2015.}\\
    % ICORR 2013
    \LCol{2013}{} & \RCol{%
        \textbf{K. R. Lyons} and S. S. Joshi,
        ``Paralyzed Subject Controls Telepresence Mobile Robot Using Novel
            {sEMG} Brain-Computer Interface: Case Study,''
        \emph{Proceedings of the IEEE International Conference on
            Rehabilitation Robotics (ICORR)},
        Seattle, WA,
        2013.}\\
\end{longtable}

\section*{Conference Posters and Abstracts}

\begin{longtable}{cc}
    % EMBC 2016 (Maria's 1-page paper)
    \LCol{2016}{} & \RCol{%
        I. M. Skavhaug, \textbf{K. R. Lyons}, S. D. Muroff, H. Chen, L. Barry,
            B. Korte, and S. S. Joshi,
        ``Fitts' Law Evaluation of a Passive Rotation Paradigm for
            Two-Dimensional Cursor Control with a Single sEMG Signal''
        \emph{Proceedings of the IEEE Engineering in Medicine and Biology Society
            Conference (EMBC)},
        Orlando, FL,
        2016.}\\
    % SFN 2015 (my poster)
    \LCol{2015}{} & \RCol{%
        \textbf{K. R. Lyons} and S. S. Joshi,
        ``Real-Time Myoelectric Control of a Virtual Upper Limb Prosthesis via
            Lower Leg Gestures: Preliminary Results,''
        \emph{Annual Meeting of the Society for Neuroscience (SfN)},
        Chicago, IL,
        2015.}\\
    % SFN 2015 (Maria's poster)
    \LCol{2015}{} & \RCol{%
        I. M. Skavhaug, \textbf{K. R. Lyons}, A. Nemchuk, S. Muroff, and S. S.
            Joshi,
        ``Control of a Cursor in Two Dimensions with One Single sEMG Signal:
            Learning of a Novel Motor Skill,''
        \emph{Annual Meeting of the Society for Neuroscience (SfN)},
        Chicago, IL,
        2015.}\\
    % SFN 2014 (my poster)
    \LCol{2014}{} & \RCol{%
        \textbf{K. R. Lyons} and S. S. Joshi,
        ``Arm Prosthetic Control through Electromyographic Recognition of Leg
            Gestures,''
        \emph{Annual Meeting of the Society for Neuroscience (SfN)},
        Washington D.C.,
        2014.}\\
    % SFN 2014 (Maria's poster)
    \LCol{2014}{} & \RCol{%
        I. M. Skavhaug, C. Dao, \textbf{K. R. Lyons}, A. Powell, L. Davidson,
            S. Joshi,
        ``Use of an Ear-Mounted Myoelectric Human-Computer Interface in the
            Home: A Pediatric Case Study with Tetra-Amelia Syndrome Subject,''
        \emph{Annual Meeting of the Society for Neuroscience (SfN)},
        Washington D.C.,
        2014.}\\
    % SFN 2014 (Walk Again a)
    \LCol{2014}{} & \RCol{%
        A. Lin, D. Schwarz, R. Sellaouti, S. Shokur, R. C. Moioli, F. L. Brasil,
            K. R. Fast, N. A. Peretti, A. Takigami, S. Gallo, \textbf{K. R.
            Lyons}, P. Mittendorfer, M. Lebedev, S. Joshi, G. Cheng, E. Morya,
            A. Rudolph, M. Nicolelis,
        ``The Walk Again Project: Brain-Controlled Exoskeleton Locomotion,''
        \emph{Annual Meeting of the Society for Neuroscience (SfN)},
        Washington D.C.,
        2014.}\\
    % SFN 2014 (Walk Again b)
    \LCol{2014}{} & \RCol{%
        F. L. Brasil, R. C. Moioli, S. Shokur, K. Fast, A. L. Lin, N. A.
            Peretti, A. Takigami, \textbf{K. R. Lyons}, D. J. Zielinski, L.
            Sawaki, S. Joshi, E. Morya, M. A. L. Nicolelis,
        ``The Walk Again Project: An EEG/EMG Training Paradigm to Control
            Locomotion,''
        \emph{Annual Meeting of the Society for Neuroscience (SfN)},
        Washington D.C.,
        2014.}
\end{longtable}

\end{document}
%80%%%%%%%%%%%%%%%%%%%%%%%%%%%%%%%%%%%%%%%%%%%%%%%%%%%%%%%%%%%%%%%%%%%%%%%%%%%%%
