%80%%%%%%%%%%%%%%%%%%%%%%%%%%%%%%%%%%%%%%%%%%%%%%%%%%%%%%%%%%%%%%%%%%%%%%%%%%%%%
% template originally found here:
%     http://texblog.org/2012/04/25/writing-a-cv-in-latex/
\documentclass[10pt]{article}

\usepackage{array, xcolor, bibentry}
\usepackage[margin=2cm]{geometry}
\usepackage{sectsty}

\sectionfont{\normalsize\mdseries\MakeUppercase}

% make a table separated by a light grey vline
\definecolor{lightgray}{gray}{0.8}
\newcolumntype{L}{>{\raggedleft}p{0.14\textwidth}}
\newcolumntype{R}{p{0.8\textwidth}}
\newcommand\VRule{\color{lightgray}\vrule width 0.5pt}


\begin{document}

\thispagestyle{empty}
\pagestyle{empty}

% title (name)
\begin{center}
{\Large Kenneth R. Lyons}
\end{center}
\vspace{1em}

% left side contact info
\begin{minipage}[ht]{0.68\textwidth}
    Robotics, Autonomous Systems, \\
    and Controls Laboratory\\
    Academic Surge 2334\\
    UC Davis\\
    Davis, CA 95616
\end{minipage}
% right side contact info
\begin{minipage}[ht]{0.5\textwidth}
    619 Pole Line Rd. Apt. 151\\
    Davis, CA 95618\\
    (530) 400-0136\\
    \texttt{ixjlyons@gmail.com}\\
    \texttt{http://ixjlyons.github.io}
\end{minipage}

\section*{Education}
\begin{tabular}{L!{\VRule}R}
    2012--present
        & {\textbf{University of California, Davis}, Davis, California}\\
        & Ph.D. in Mechanical and Aerospace Engineering. (expected June 2017)\\
        & Advisor: Sanjay S. Joshi\\
    [5pt]
    2008--20012
        & {\textbf{University of California, Davis}, Davis, California}\\
        & B.S. in Mechanical Engineering. Graduation with high honors. Minor in
            Linguistics.
\end{tabular}


\section*{Awards and Recognition}
\begin{tabular}{L!{\VRule}R}
    2012
        & UC Davis Department of Mechanical and Aerospace Engineering Service
            Award, for participation in the Computing and Robotics Outreach
            Club\\
    [5pt]
    2012
        & UC Davis Clinical and Translational Science Center / College of
            Engineering Summer Camp Program\\
\end{tabular}


\section*{Experience}
\begin{tabular}{L!{\VRule}R}
    6/12--present
        & {\bf Graduate student researcher} Robotics, Autonomous Systems, and
            Controls Laboratory, UC Davis\\
        & Work on computer and machine interface control using electromyography,
            including upper limb prosthetic control. Lead Android application
            programmer for additional laboratory experiments.\\
    [5pt]
    9/12--12/12
        & {\bf Teaching Assistant}, UC Davis\\
        & Supervised laboratory sessions in experimental methods course (EME
            107B) and graded lab reports.\\
    [5pt]
    1/12--4/12
        & {\bf Undergraduate Student Researcher}, Sports Biomechanics
            Laboratory, UC Davis\\
        & Assisted Ph.D. student with riderless bicycle project, including some
            embedded system analysis and literature review of bicycle
            dynamics.\\
    [5pt]
    9/10--6/12
        & {\bf Tutor at Superb Tutors}\\
        & Tutored high school and college students in math, physics, and
            engineering courses.\\
\end{tabular}


\section*{Projects}
\begin{tabular}{L!{\VRule}R}
    2014--present
        & {\bf pygesture}\\
        & Open-source myoelectric gesture recognition suite for end-to-end
            prosthesis control experiments, written in Python. Includes data
            acquisition, signal processing, classification, graphical user
            interface, and communication with real-time simulation software.\\
        & \texttt{http://github.com/ixjlyons/pygesture}\\
    [5pt]
    2013--2014
        & {\bf Walk Again}\\
        & International project which produced a brain-controlled exoskeleton
            demonstrated at the 2014 FIFA World Cup. Worked as a part of the
            human-machine interface team and created an LED-based feedback
            system to enable robust control during the demonstration.\\
\end{tabular}


\section*{Publications}
\begin{tabular}{L!{\VRule}R}
    % Human Movement Science
    2015 &
        I. M. Skavhaug, \textbf{K. R. Lyons}, A. Nemchuk, S. Muroff, and S.
            Joshi,
        ``Two-Dimensional Cursor Control from a Single-Site Surface
            Electromyography Signal: Characterization of Learning,''
        \emph{Human Movement Science},
        (in revision).\\
    % RESNA 2015
    2015 &
        \textbf{K. R. Lyons} and S. S. Joshi,
        ``A Case Study on Classification of Foot Gestures via Surface
            Electromyography,''
        \emph{Rehabilitation Engineering and Assistive Technology Society of
            America (RESNA) Annual Conference},
        Denver, CO,
        2015.\\
    [5pt]
    % ICORR 2013
    2013 &
        \textbf{K. R. Lyons} and S. S. Joshi,
        ``Paralyzed Subject Controls Telepresence Mobile Robot Using Novel
            {sEMG} Brain-Computer Interface: Case Study,''
        \emph{Rehabilitation Robotics (ICORR), 2013 IEEE International
            Conference on},
        Seattle, WA,
        2013.\\
\end{tabular}


\section*{Conference Posters and Abstracts}
\begin{tabular}{L!{\VRule}R}
    % SFN 2015 (my poster)
    2015 &
        \textbf{K. R. Lyons} and S. S. Joshi,
        ``Real-Time Myoelectric Control of a Virtual Upper Limb Prosthesis via
            Lower Leg Gestures: Preliminary Results,''
        \emph{Annual Meeting of the Society for Neuroscience (SfN)},
        Chicago, IL,
        2015.\\
    [5pt]
    % SFN 2015 (Maria's poster)
    2015 &
        I. M. Skavhaug, \textbf{K. R. Lyons}, A. Nemchuk, S. Muroff, and S. S. Joshi,
        ``Control of a Cursor in Two Dimensions with One Single sEMG Signal:
            Learning of a Novel Motor Skill,''
        \emph{Annual Meeting of the Society for Neuroscience (SfN)},
        Chicago, IL,
        2015.\\
    [5pt]
    % SFN 2014 (my poster)
    2014 &
        \textbf{K. R. Lyons} and S. S. Joshi,
        ``Arm Prosthetic Control through Electromyographic Recognition of Leg
            Gestures,''
        \emph{Annual Meeting of the Society for Neuroscience (SfN)},
        Washington D.C.,
        2014.\\
    [5pt]
    % SFN 2014 (Maria's poster)
    2014 &
        I. M. Skavhaug, C. Dao, \textbf{K. R. Lyons}, A. Powell, L. Davidson,
            S. Joshi,
        ``Use of an Ear-Mounted Myoelectric Human-Computer Interface in the
            Home: A Pediatric Case Study with Tetra-Amelia Syndrome Subject,''
        \emph{Annual Meeting of the Society for Neuroscience (SfN)},
        Washington D.C.,
        2014.\\
    [5pt]
    % SFN 2014 (Walk Again a)
    2014 &
        A. Lin, D. Schwarz, R. Sellaouti, S. Shokur, R. C. Moioli, F. L. Brasil,
            K. R. Fast, N. A. Peretti, A. Takigami, S. Gallo, \textbf{K. R.
            Lyons}, P. Mittendorfer, M. Lebedev, S. Joshi, G. Cheng, E. Morya,
            A. Rudolph, M. Nicolelis,
        ``The Walk Again Project: Brain-Controlled Exoskeleton Locomotion,''
        \emph{Annual Meeting of the Society for Neuroscience (SfN)},
        Washington D.C.,
        2014.\\
    [5pt]
    % SFN 2014 (Walk Again b)
    2014 &
        F. L. Brasil, R. C. Moioli, S. Shokur, K. Fast, A. L. Lin, N. A.
            Peretti, A. Takigami, \textbf{K. R. Lyons}, D. J. Zielinski, L.
            Sawaki, S. Joshi, E. Morya, M. A. L. Nicolelis,
        ``The Walk Again Project: An EEG/EMG Training Paradigm to Control
            Locomotion,''
        \emph{Annual Meeting of the Society for Neuroscience (SfN)},
        Washington D.C.,
        2014.\\
\end{tabular}

\end{document}
%80%%%%%%%%%%%%%%%%%%%%%%%%%%%%%%%%%%%%%%%%%%%%%%%%%%%%%%%%%%%%%%%%%%%%%%%%%%%%%
